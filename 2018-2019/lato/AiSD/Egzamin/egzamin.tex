\documentclass[12pt]{article}
\usepackage[utf8]{inputenc}
\usepackage[T1]{fontenc} 
\usepackage[polish]{babel}
\usepackage{amsmath}
\usepackage{amsthm}
\usepackage[margin=0.7in]{geometry}
\usepackage{dsfont}
\usepackage{mdframed}
\usepackage{titlesec}
\usepackage{gensymb}

\begin{document}
\begin{titlepage}

\title{\bfseries Algorytmy i Struktury Danych\\\Large Egzamin - Pierwszy termin}
\date{\today}
\author{JG}
\end{titlepage}

\maketitle

\begin{enumerate}

    \item{Czy usunięcie wierzchołka z drzewa AVL może wymagać 3 rotacji
    dla przywrócenia balansu? Narysuj przykład albo napisz uzasadnienie
    dlaczego nie.}
    
    \item{W algorytmie \textit{LazySelect} znajdującym medianę ciągu $C$
    wyznaczana jest losowa próbka $H$. W tym celu $n^{3/4}$ razy losujemy (ze zwracaniem)
    elementy ciągu $C$. Załóżmy, że $C$ składa się z $n$ różnych elementów. Co się
    stanie jeśli za każdym razem do $H$ został wylosowany ten sam element?
    Co jeśli do $H$ trafiły dokładnie dwa różne elementy?}
    
    \item{Tworząc słownik statyczny (dwupoziomowy) losujemy funkcję $h$ z
    rodziny uniwersalnej i rozrzucamy $n$ kluczy do tablicy $n$-elementowej. Co jeżeli 
    do pewnego kubełka, trafi $\sqrt{n}$ kluczy?}
    
    \item{Napisać \textit{InsertSort} w pseudokodzie.}
    
    \item{Podać (jak najdokładniejsze) asymptotyczne ograniczenie na głębokość sieci
        przełączników realizujących wszystkie przesunięcia cykliczne ciągu wejściowego}

    \item{W której wersji \textit{deletemin} na kopcu spodziewamy się wykonać
    mniej porównań i dlaczego? (usuwany element zastępujemy skrajnie prawym liściem
    z ostatniego poziomu \textbf{vs} przesuwanie dziury na dół).}

    \item{Napisać procedurę wstawiania elementu do struktury van Emde Boasa w 
    pseudokodzie i napisać jej złożoność z uzasadnieniem.}

    \item{Ile elementów może liczyć zbiór $\{w^2 | \,\,w$ jest 
    $n$-tym pierwiastkiem zespolonym z jedności $\}$}? Odpowiedź uzasadnij.

    \item{Podany na wykładzie równoległy algorytm mnożenia dwóch liczb redukuje
    dodawanie trzech liczb do dodawania dwóch liczb. Zmodyfikuj ten algorytm 
    tak, by redukował dodawanie siedmiu liczb do dodawania trzech liczb. Jaką
    złożoność będzie miał ten algorytm?}

    \item{Jaka jest maksymalna liczba rotacji podczas pojedynczej operacji
    słownikowej na drzewach \textit{Splay} o $n$ wierzchołkach?}

    \item{Jaka jest oczekiwana liczba kolizji gdy do umieszczenia 30 różnych
    kluczy w tablicy 300 elementowej użyjemy funkcji losowo wybranej z
    uniwersalnej rodziny funkcji haszujących? Odpowiedź uzasadnij.}

    \item{Czy wysokość drzew powstałych w kopcu Fibonacciego o $n$ wierzchołkach
    da się ograniczyć przez $log^2n$?}

    \item{Rozważamy haszowanie metodą adresowania otwartego, w której konflikty
    rozwiązujemy metodą kwadratową. Czy funkcja $h(k,i)=(h'(k)+100i+1000i^2)\,\,mod\,\, 10000$}
    jest sensownym wyborem?

    \item{Udowodnij, że jeśli po ''przepuszczeniu'' ciągu bitonicznego $a_1,...,a_{2n}$
    przez sieć półczyszczącą otrzymamy ciąg $b_1,...,b_{2n}$ to $b_1,...,b_{n}$
    oraz $b_{n+1},...,b_{2n}$ są ciągami bitonicznymi.}
    
    \item{Zauważyłeś, że \textit{Quicksort} (z deterministycznym pivotem) zachowuje
    się zadziwiająco regularnie na ciągach z pewnej rodziny $A$. Otóż okazało
    się, że w trakcie wszystkich wywołań rekurencyjnych proderura \textit{Partition}
    dokonuje podziału ciągu wejściowego na podciągi o długościach nie mniejszych
    niż 1/3 i nie większych niż 2/3 długości ciągu wejściowego. W jakim
    czasie działa Quicksort na ciągach z rodziny $A$?}

    \item{Algorytm Boruvki-Sollina wykonuje pewną liczbę faz. W każdej fazie
    dorzuca do konstruowanego drzewa pewne krawędzie i tworzy nowy graf, który będzie rozważany
    w kolejnej fazie. Skonstruuj (jak najprostszy) graf, dla którego ten algorytm
    wykona dokładnie 3 fazy lub pokaż, żę taki graf nie istnieje.}

    \item{W analizie problemu \textit{Union-Find} wykorzystywaliśmy pojęcia rzędu
    wierzchołka oraz grupy rzędu. Przypomnij definicje tych pojęć. Ile maksymalnie
    bitów potrzebujemy przeznaczyć na pamiętanie rzędu w każdym wierzchołku?}

    \item{Oblicz funkcję $\pi$ dla wzorca \texttt{aabaaabaaaba}.}

    \item{Udowodnij, że przynajmniej jedna z operacji \textit{min, deletemin} lub \textit{insert}
    wykonywanych na kolejce priorytetowej wymaga w najgorszym przypadku czasu $\Omega(log\,n)$}

    \item{W algorytmie \textit{Shift-And} wykonywane są operacje logiczne na
    słowach maszynowych. Wytłumacz w jaki sposób?}
\end{enumerate}

\end{document}



