\documentclass[12pt]{article}
\usepackage[utf8]{inputenc}
\usepackage[T1]{fontenc}
\usepackage[polish]{babel}
\usepackage{amsmath}
\usepackage{amsthm}
\usepackage[margin=0.7in]{geometry}
\usepackage{dsfont}
\usepackage[many]{tcolorbox}
\usepackage{listings}
\usepackage[bottom]{footmisc}

\begin{document}
\begin{titlepage}

\title{\bfseries Algorytmy i Struktury Danych\\\Large Egzamin - część druga}
\date{\today}
\author{JG}
\end{titlepage}

\maketitle

\begin{enumerate}

    \item Skonstruuj efektywny algorytm, który dla danego ciągu
    znajdzie jego najdłuższy podciąg, który jest palindromem.

    \item Drzewem wyprowadzania słowa $w$ w gramatyce bezkontekstowej
    jest drzewo $T$ etykietowane literami alfabetu, takie, że
    \begin{itemize}
        \item etykietą korzenia jest symbol początkowy
        \item dla każdego wierzchołka wewnętrznego $v$: jeśli
        etykietą $v$ jest $A$ a etykiety dzieci $v$ czytane z lewej
        na prawo tworzą słowo $u$, to $A \rightarrow u$ jest produkcją
        w gramatyce.
        \item etykiety liści czytane z lewej na prawo tworzą słowo $w$
    \end{itemize}
    Głębokością wyprowadzenia słowa $w$ w gramatyce $G$ nazywamy najmniejszą
    wysokość drzewa wyprowadzenia dla słowa $w$. Ułóż algorytm, który dla słowa
    $w$ oblicza głębokość jego wyprowadzenia w gramatyce $G$ w postaci Chomsky'ego.

    \item Dany jest zbiór $A$ punktów na okręgu oraz funkcja 
    $c: A \times A \rightarrow R_+$ określająca wagi odcinków pomiędzy
    punktami ze zbioru $A$. Zakładamy, że $A$ zawiera $2n$ punktów.
    \begin{itemize}
        \item Ile jest różnych zbiorów zlożonych z $n$ nieprzecinających
        się odcinków łączących punkty z $A$?
        \item Ułóż algorytm znajdujący zbiór takich odcinków o minimalnej wadze.
    \end{itemize}
\end{enumerate}

\end{document}



