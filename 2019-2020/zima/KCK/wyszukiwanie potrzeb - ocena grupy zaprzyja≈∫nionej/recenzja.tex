\documentclass[12pt, a4paper, oneside]{article}

\usepackage[utf8]{inputenc}
\usepackage[T1]{fontenc}
\usepackage[polish]{babel}


\title{Recenzja pracy\\ \textit{,,Wybory przyjazne każdemu obywatelowi''}\\
       \large(zadanie trzecie -- znajdowanie potrzeb)}
\date{}
\author{
    \textbf{Zaprzyjaźniona grupa oceniana \small(czwartkowa)}:
        \\ Daniela Tataruda
        \\ Dawid Paluszak
        \\ Patryk Wilusz\\
    \\\textbf{Grupa oceniająca \small(czwartkowa)}:
        \\ Jakub Remiszewski
        \\ Kacper Bukowiec
        \\ Jakub Grobelny
}

\begin{document}

\begin{titlepage}
    \maketitle
\end{titlepage}

\section*{Wstęp}

W niniejszej recenzji przyjrzymy się pracy pt. ,,\textit{Wybory przyjazne 
każdemu obywatelowi}'' autorstwa zaprzyjaźnionej grupy. Ocenimy jej 
poszczególne rozdziały pod kątem zastosowanej metody badawczej i słuszności 
wniosków wyciągniętych z badania. 

\section*{Ocena}

\subsection*{\textit{,,1. Wstęp''}}

Wstęp do pracy dobrze wyjaśnia jakie były motywacje do przeprowadzenia badania. 
Aktualny system przeprowadzania wyborów poddany został słusznej krytyce, ale 
można odnieść wrażenie, iż autorzy podeszli do zadania ze z góry założoną tezą. 
Widać że mają na opisywany temat dość silne opinie, które mogą budzić 
wątpliwości co do obiektywności badania.

\subsection*{\textit{,,2. Ankieta''}}

Ankieta zadana uczestnikom badania składa się z 6 pytań. Wydaje się, że to dość 
mało, ale poruszają one wszystkie kwestie istotne do ustalenia, czy możliwość 
oddawania głosów elektronicznie jest dla społeczeństwa istotna, i czy na takiej 
zmianie skorzystałoby wiele osób. Podsumowując, widać, że pytania do ankiety zostały dobrze przemyślane i wybrane starannie.

\subsection*{\textit{,,3. Wyniki ankiety i obserwacje''}}

Autorzy przytaczają odpowiedzi trzech wybranych osób, których wypowiedzi były 
(według nich) najbardziej trafne. Niestety nie doprecyzowali oni, co to 
dokładnie oznacza. Nie budzi natomiast żadnych zastrzeżeń opis wybranych 
badanych, który jest precyzyjny i wyczerpujący (w sensie związków badanych z poruszanym problemem) bez naruszania jednocześnie ich anonimowości. 
Autorzy skrupulatnie wymienili odpowiedzi poszczególnych osób na pytania z 
ankiety.

\subsection*{\textit{,,4. Potrzeby''}}

Na podstawie ankiety autorzy przedstawili dziesięć potrzeb, jakie należy 
spełnić w celu usprawnienia systemu głosowania. Większość z nich jest sensowna 
i wynika z tego, co odpowiedzieli (wymienieni) badani.

\subsection*{\textit{,,5. Propozycje ulepszeń''}}

Zaprzyjaźniona grupa podała również pięć ,,propozycji ulepszeń''. Piąta z nich (dokładniejsze i rzetelniejsze statystyki) jest jednak bardziej potrzebą lub opisem  pozytywnych skutków wprowadzenia głosowania przez internet. Wspomniana 
zostaje również ,,aplikacja'', o której autorzy niestety nic wcześniej nie napisali, więc nie wiadomo jaka miałaby być jej dokładna funkcjonalność. Proponowane wprowadzenie pomocy przy głosowaniu dla osób starszych wydaje się być dość kosztownym przedsięwzięciem.

\section*{Podsumowanie}

Podsumowując, oceniana grupa wykonała swoje zadanie co najmniej poprawnie. 
Jedyne poważniejsze wady recenzowanej pracy znajdują się, naszym zdaniem, w 
dziale ,,propozycje ulepszeń''. Poza tym badanie potrzeb zostało przeprowadzone 
w sposób rzetelny i staranny. W skali od 0 do 10 autorom należy się, według nas,
ocena przynajmniej 8.

\end{document}