\documentclass[12pt]{article}
\usepackage[utf8]{inputenc}
\usepackage[T1]{fontenc}
\usepackage[polish]{babel}
\usepackage{amsmath}
\usepackage{amsthm}
\usepackage{dsfont}
\usepackage{abstract}
\usepackage{titlesec}
\usepackage{algorithmicx}
\usepackage{algpseudocode}

\titleformat{\section}[block]{\Large\bfseries\filcenter}{}{0em}{}
\newtheorem*{theorem*}{Teza}
\newtheorem*{proof*}{Dowód}

\title{\bfseries Podstawy i Zastosowania Złożoności Obliczeniowej\\\Large Zadanie 1}
\date{}
\author{\Large Jakub Grobelny}

\begin{document}

\begin{titlepage}
\maketitle
\thispagestyle{empty}

\hrulefill
\section{Podpunkt a}
\begin{theorem*}\,\\
$3COL \leq_P Tutorzy$
    \begin{proof*} \,\\\normalfont    
    Pokażemy, że istnieje obliczalna w czasie wielomianowym funkcja $f$ taka, 
    że $\forall_G(G \in 3COL \Leftrightarrow f(G) \in Tutorzy)$.
    
\begin{verbatim}
procedura f(G(V, E)):
    przemianuj(G)
    konflikty := []
    studenci := |V| + 1
    dla każdego v z V:
        dodaj {"zrzeda": |V| + 1, "nielubiany": v} do konflikty
        dla każdego w sąsiadującego z v:
            dodaj {"zrzeda": v, "nielubiany": w} do konflikty
    zwróć {"studenci": studenci, "konflikty": konflikty}

procedura przemianuj(G(V, E)):
    dla każdego i z 1...|V|:
        zmień nazwę wierzchołka V[i] na i

\end{verbatim}

Funkcja $f$ obliczana przez powyższy program zamienia dany graf $G$ na 
instancję problemu $Tutorzy$ poprzez zamienienie każdego wierzchołka na 
studenta. Każdy wierzchołek $v$, od którego w grafie $G$ wychodzi krawędź 
incydentna do wierzchołka $w$, staje się \textit{zrzędą} $v$, która nie chce 
mieć tego samego tutora co student $w$. Dodatkowo tworzymy jednego sztucznego 
studenta o numerze $|V|+1$, który nie lubi wszystkich innych studentów. Na początku wywołujemy procedurę \texttt{przemianuj}, która sprawia, że wszystkie wierzchołki mają nazwy będące liczbami z odpowiedniego zakresu.

Widać, że program działa w czasie $O(|V|\cdot|E|+T(G))$, gdzie 
$T(G)$ to czas działania procedury \texttt{przemianuj}, gdyż każdy 
wierzchołek grafu odwiedzamy jeden raz, a dla każdego wierzchołka sprawdzamy 
jego sąsiadów również tylko raz (w najgorszym przypadku każdy wierzchołek 
sąsiaduje z każdym). Dodawanie obiektów do listy można zrealizować w czasie 
stałym a pomocnicza procedura \texttt{przemianuj} działa w czasie 
wielomianowym. Funkcja $f$ jest zatem obliczalna w czasie wielomianowym
względem rozmiaru danego grafu.

$ $\\
Pokażemy teraz, że $\forall_G(G \in 3COL \Leftrightarrow f(G) \in Tutorzy)$:

$ $\\
$\Rightarrow$ 

Weźmy dowolny graf $G \in 3COL$. Skoro $G$ jest 3-kolorowalny, to jeżeli dodamy 
do niego nowy wierzchołek $v$, taki że sąsiaduje z każdym innym 
istniejącym już wierzchołkiem, to graf ten będzie 4-kolorowalny (wystarczy 
wziąć dowolne 3-kolorowanie $G$ a nowy wierzchołek pomalować na nowy, czwarty 
kolor). Niech $G^+$ będzie grafem $G$ z dodanym wyżej opisanym wierzchołkiem $v$.
Wówczas każdy wierzchołek z $G$ ma odpowiadającego mu studenta w $f(G)$, 
zaś dodatkowy wierzchołek $v$ odpowiada sztucznie dodanemu studentowi, który 
nie lubi nikogo. Każda krawędź z $G^+$ jest reprezentowana przez 
\textit{konflikty}. 

Wprowadźmy bijekcję $g : Kolor \rightarrow Tutor$, która odwzorowuje kolory na tutorów.
Każdemu studentowi $s$ możemy przydzielić tutora $g(c)$, gdzie $c$ jest 
kolorem wierzchołka $w$ z $G^+$ odpowiadającego studentowi $s$. Skoro $G^+$ 
jest 4-kolorowalny, a $g$ jest bijekcją, to żaden student z $f(G)$ nie będzie w grupie ze studentem, którego nie lubi, czyli $f(G) \in Tutorzy$.

$ $\\
$\Leftarrow$

Weźmy dowolny graf $G$ taki, że $f(G) \in Tutorzy$. Pokażemy, że $G \in 3COL$.


Niech \texttt{T} $= f(G)$ i niech $H$ będzie grafem o poniższej konstrukcji:
\begin{itemize}
    \item Zbiór wierzchołków grafu $H$ to \{1, 2,\,..., \texttt{T.studenci}\}
    \item Dla dowolnych $v$ i $w$, jeżeli do \texttt{T.konflikty} należy \\
        \texttt{\{"\hskip0pt zrzeda":} $v$\texttt{, "\hskip0pt nielubiany":} 
        $w$ \texttt{\}}, to wówczas w $H$ istnieje krawędź pomiędzy $v$ a $w$.
\end{itemize}

$H$ jest w oczywisty sposób 4-kolorowalny, gdyż wybór jednego spośród czterech 
tutorów odpowiada wyborowi jednego spośród czterech kolorów, a skoro krawędzie 
w grafie $H$ opisują relację \textit{konflikty} z obiektu \texttt{T}, to 
wówczas mamy gwarancję, że dwa wierzchołki o tym samym kolorze (tutorze) nie 
sąsiadują ze sobą w grafie $H$.

Z definicji $f$ wiemy również, że w \texttt{T} istnieje student $s$, który jest \textit{zrzędą} i nie lubi wszystkich innych studentów. W takim razie w $H$ istnieje wierzchołek $v$, taki że jest połączony z każdym innym wierzchołkiem w tym grafie. Skoro $f(G) \in Tutorzy$, to student $s$ jest jedynym studentem przypisanym do swojego tutora $t$, bo w przeciwnym razie byłby w grupie razem z kimś, kogo nie lubi. Z tego wynika, że w 4-kolorowaniu grafu $H$ odpowiadającemu przydziałowi tutorów, wierzchołek $v$ jako jedyny ma swój kolor $c$. Niech $H^-$ będzie grafem powstałym poprzez usunięcie wierzchołka $v$ oraz incydentnych do niego krawędzi z grafu $H$. W oczywisty sposób $H^-$ jest 3-kolorowalny.

Możemy zauważyć, że $\forall_{G', H'}(f(G') = f(H') \Leftrightarrow G'$ jest 
izomorficzny z $H'$). Własność ta wynika z tego, że pomocnicza procedura 
\texttt{przemianuj} przeetykietowała jedynie wierzchołki (graf otrzymany po 
zaaplikowaniu \texttt{przemianuj} do grafu jest izomorficzny z oryginalnym) a 
dalsza część przekształcenia $f$ jest w pełni odwracalna. Widać, że $f(G) 
= f(H^-)$, bo $H^-$ ma dokładnie takie same krawędzie jak $G$ (z dokładnością 
do nazw wierzchołków) oraz tyle samo wierzchołków. W takim razie grafy $H^-$ oraz $G$ są izomorficzne. Zaaplikowanie izomorfizmu nie zmienia kolorowalności grafu, więc $G$ jest 3-kolorowalny, czyli $G \in 3COL$.
\begin{flushright}
    $\qedsymbol$
\end{flushright}

\end{proof*} \end{theorem*}

\hrulefill

\section{Podpunkt b}

\begin{theorem*}\,\\\normalfont
Problem $Tutorzy$ można rozwiązać w czasie wielomianowym, przy założeniu, że 
będzie co najwyżej 15 zrzęd.

\begin{proof*}\,\\\normalfont

Pokażemy, że dla dowolnej stałej liczby zrzęd $K$ istnieje algorytm działający 
w czasie wielomianowym, który rozwiązuje problem $Tutorzy$.

\begin{verbatim}
procedura tutorzy(S):
    S' := S bez studentów, którzy nie są zrzędami
    P  := wszystkie poprawne przydziały tutorów dla S'
    jeżeli P jest puste:
        zwróć NIE
    dla każdego p z P:
        dla każdego studenta s z S, który nie jest zrzędą:
            T := tutorzy przydzieleni zrzędom nie lubiącym s
            jeżeli |T| = 4:
                wyjdź z pętli i rozpatrz kolejne p
        zwróć TAK
    zwróć NIE
\end{verbatim}

$ $\\
Zasada działania:

Algorytm najpierw przydziela tutorów dla podzbioru 
studentów, którzy są  zrzędami (dowolnym algorytmem rozwiązującym ogólny 
problem $Tutorzy$). Jeżeli nie da się przydzielić tutorow 
dla samych zrzęd, to wówczas można od razu 
zwrócić odpowiedź ,,\texttt{NIE}'', gdyż tym bardziej nie można wtedy 
przydzielić tutorów reszcie studentów. Następnie rozpatrujemy wszystkie 
poprawne przydziały tutorów dla zrzęd. Żeby dokończyć przydział tutorów, musimy 
wybrać tutora dla każdego pozostałego studenta. Można zauważyć, że jedyni 
studenci, jakich trzeba rozważyć przy doborze tutora, to zrzędy. Algorytm 
zlicza wszystkich unikalnych tutorów jacy zostali przydzieleni zrzędom, które 
nie lubią aktualnie rozpatrywanego studenta. Jeżeli jest to czterech różnych 
tutorów, to wówczas nie ma żadnego dostępnego dla rozpatrywanego studenta. W 
takim przypadku wychodzimy z wewnętrznej pętli i rozpatrujemy kolejny przydział 
\texttt{p}. Jeżeli okaże się, że w wewnętrznej pętli udało się znaleźć tutora dla każdego studenta, to zwracamy odpowiedź ,,\texttt{TAK}''. Jeżeli rozpatrzymy bez sukcesu wszystkie przydziały \texttt{p}, to wówczas zwracamy odpowiedź ,,\texttt{NIE}''.

$ $\\
$ $\\
$ $\\
Analiza złożoności poszczególnych kroków:

\begin{enumerate}
    \item Przefiltrowanie studentów aby uzyskać instancję \texttt{S'} problemu 
    można wykonać w czasie wielomianowym. Wystarczy wziąć listę konfliktów z 
    \texttt{S}, usunąć te konflikty, gdzie zrzęda nie lubi studenta, który nie 
    jest zrzędą a następnie wyznaczyć nowe numery dla studentów aby leżały w 
    odpowiednim przedziale.
    \item Przydziały \texttt{P} możemy wyznaczyć dowolnym algorytmem 
    rozwiązującym problem $Tutorzy$, ponieważ w \texttt{S} jest nie więcej 
    niż $K$ studentów (a $K$ jest ustaloną stałą), więc zajmuje to czas $O(1)$.
    \item \texttt{|P|} jest stałą (być może bardzo dużą) ze względu na stałą 
    liczbą studentów. W związku z tym, zewnętrzna pętla wykona się stałą liczbę 
    razy.
    \item |\texttt{S} $\setminus$ \texttt{S'}| = $O$(\texttt{S.studenci}), więc 
    pętla wykona się liniową liczbę razy względem rozmiaru wejścia.
    \item Wyznaczenie \texttt{T} zajmuje najwyżej czas $K\cdot$ \texttt{S.studenci} (może być w czasie stałym jeżeli przed rozpoczęciem pętli stworzylibyśmy macierz sąsiedztwa studentów, w której zapisana byłaby informacja o tym, przez jakich studentów dany student jest nielubiany).
\end{enumerate}

Sumarycznie czas potrzebny na wykonanie programu \texttt{tutorzy(S)} jest 
wielomianowy.

$ $\\
Pokazaliśmy, że istnieje algorytm rozwiązujący problem $Tutorzy$ z 
ograniczeniem do $K$ zrzęd w czasie wielomianowym. Jeżeli istnieje taki 
algorytm dla dowolnej stałej $K$, to w szczególności da się rozwiązać taki 
problem w czasie wielomianowym przy założeniu, że będzie co najwyżej 15 zrzęd, 
co należało udowodnić.
\begin{flushright}
    $\qedsymbol$
\end{flushright}

\end{proof*} \end{theorem*}

\end{titlepage}
\end{document}
